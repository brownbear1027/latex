\documentclass[12pt,a4paper,openany]{book}
\usepackage[a4paper,top=35mm,bottom=35mm,left=25mm,
            right=25mm,marginparwidth=1.75cm]{geometry}

\usepackage{amsmath,amsthm,amssymb,amstext}
\usepackage{enumitem}
\setlist[enumerate]{nosep,labelindent=\parindent,
label=\arabic*),
leftmargin=*,parsep=2pt,itemsep=2pt}
\usepackage{mdframed}

\setcounter{secnumdepth}{0}
\setcounter{tocdepth}{0}

\parindent=0pt
\parskip=4pt

\begin{document}

\chapter*{CHAPTER 1}
 
\section{\underbar{DEFINITION OF STATISTICS}}

\textbf{Statistics} is the science of conducting 
studies to collect, organize, summarize, analyze and 
draw conclusions from data.

\subsection{\underbar{Descriptive and Inferential Statistics}}

\textbf{A variable} is a characteristic or attribute 
that can assume different values.

\textbf{Data }are the values (measurements or observations) 
that are variables can assume. Variables whose values are 
determined by chance are called \textbf{random variables}.

A collection of data values forms a \textbf{data set}. 
Each value in the data set is called a \textbf{data} 
value or a \textbf{datum}.

Data can be used in different ways. Statistics can be 
divided into two main areas depending on how data are used. 
The two areas are \textbf{Descriptive Statistics} 
and \textbf{Inferential Statistics}.

\textbf{Descriptive Statistics} consists of the collection, 
organization, summation and presentation of data.

\textbf{Example:}
\begin{enumerate}
\item Nine out of ten on the job fatalities are men.
\item Expenditure for the cable industry were 5.66 
      dollar billion in 1996.
\item The median house hold income for people aged 2534 
      is 35,888 dollars.
\item The national average annual medicine 
      expenditure per person is 1052 dollar.
\end{enumerate}

\begin{mdframed}
A \textbf{population} consists of all subjects 
(human or otherwise) that are being studied.
\\
A \textbf{sample} is a subgroup of the population.
\end{mdframed}

\textbf{Inferential Statistics} consists of generalizing 
from sample to populations, performing hypothesis testing, 
determining relationship among variables and making predictions.

\textbf{Example:}
\begin{enumerate}
\item By 2040 at least 3.5 billion people will run 
      short of water.
\item Experts say that mortgage rates may soon hit bottom.
\item A diet high in fruits and vegetables will lower 
      blood pressure.
\item In 2030, the number of high school graduates 
      will be 3.2 million students.
\end{enumerate}

\subsection{\underbar{Variables and Types of Pata}}

Variables can be classified as quantitative or qualitative

\begin{mdframed}
\textbf{Qualitative variables }are variables that 
can be placed into distinct categories,
according to some characteristics or attribute.
\end{mdframed}

\textbf{Example:}
\begin{enumerate}
\item Marital status of nurses in a hospital.
\item Colours of automobiles in a shopping centre 
parking lot.
\end{enumerate}

\begin{mdframed}
\textbf{Quantitative variables} are numerical in nature 
and can be ordered or ranked.
\end{mdframed}

\textbf{Example:}
\begin{enumerate}
\item Time it takes to run a marathon
\item capacity of the NFL football stadium
\item  Ages of people living in a personal care home
\end{enumerate}

\begin{mdframed}
\textbf{Discrete variables} assume values that can be 
counted.
\end{mdframed}

\textbf{Example:}
\begin{enumerate}
\item Number of cups of coffee served in a restaurant.
\item The number of ads on a one-hour television show.
\item Number of pizzas sold by Pizza Express each day.
\end{enumerate}

\begin{mdframed}
\textbf{Continuous variables }can assume all values 
between any two specific values.

They are obtained by measuring.
\end{mdframed}

\textbf{Example:}
\begin{enumerate}
\item The time it takes a student to drive to school.
\end{enumerate}

\end{document} 